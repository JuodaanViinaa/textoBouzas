\documentclass[a4paper,12pt]{article}
\usepackage[utf8]{inputenc}
\usepackage[T1]{fontenc}
\usepackage{csquotes}
\usepackage{anysize}
\usepackage{graphicx}
\marginsize{25mm}{25mm}{25mm}{25mm}
\usepackage[colorlinks, citecolor=blue]{hyperref}
\usepackage[backend=biber,style=apa]{biblatex}
\addbibresource{~/Documents/Doctorado/library.bib}

\title{The effect of the usefulness of information on rats' preference in the suboptimal choice procedure}
\author{Daniel Maldonado}
\date{2023}

\begin{document}
{\bfseries \maketitle}

Suboptimal choice refers to both a phenomenon and a procedure in which several species---including pigeons (\cite{Gipson2009}), starlings (\cite{Vasconcelos2015}), rhesus macaques (\cite{Smith2017}), and, to an extent, humans (\cite{Molet2012})---seem to prefer an alternative that provides a relatively small amount of food over another which provides a greater amount---as long as the low-density alternative has signals that reliably predict the availability of reinforcement on each trial and the high-density alternative does not.
However, rats do not seem to share the same tendency: they prefer the high-density alternative regardless of the signaling condition. Much theoretical interest has arised aiming to explain this difference.

% \begin{figure}[!ht]
%     \begin{center}
%         \includegraphics[scale=1]{~/Documents/Doctorado/Proyecto de Doctorado/EstadoArte/Figures/gipson2009.pdf}
%     \end{center}
% \end{figure}

Several explanations for this phenomenon have been proposed ranging from local variables such as a contrast effect (\cite{Zentall2016}), differences in incentive salience attribution (\cite{Chow2017}) or conditioned inhibition ({\itshape cf.} \cite{Laude2014a,Trujano2016}), to foraging analyses (\cite{Vasconcelos2015}) and mathematical models (\cite{Cunningham2018,Daniels2018}).

This project is concerned with the ecological model put forward by Vasconcelos (\citeyear{Vasconcelos2015}). Although this explanation was proposed when only the positive results from pigeons and other species were known, it can also accommodate rats' behavior.

Vasconcelos' model stems from optimal foraging theory and proposes that, in their natural environment, pigeons---and likely other species---were never forced to pay the opportunity cost of waiting for a prey that was unavailable. That is, if they received information that signaled that a prey item was sure to escape, they would simply redirect their effort elsewhere instead of waiting around for a prey that would never come. This meant that the time associated with fruitless chases was never experienced and, in consequence, never became a part of their decision-making processes. What this implies is that modern-day pigeons ignore the time spent in the presence of negative information---whether they are in a natural environment or in the laboratory---. In the suboptimal choice procedure animals are forced to pay the opportunity cost of waiting during unreinforced terminal links, but their decision-making mechanisms are just the same as in their natural environment. Thus, they ignore time spent in the presence of negative stimuli, and this throws off the calculations for the reinforcement density of each alternative, resulting in choices which we would classify as irrational. However, it is worth noting that this suboptimality arises from a mismatch between the domain of selection and the domain of testing, and not from actual errors in their decision-making processes.

This project aims to test this proposal by making the laboratory environment more similar to the natural one. One way to do this is by making information useful again: in natural environments organisms can use information about ulterior reinforcement to alter their behavior and increase the reinforcement rate they receive. A way to integrate this possibility in the laboratory is by adding a way for organisms to end fruitless trials before their scheduled time is up. This can be done by way of an ``escape'' response which has the function of prematurely ending a trial and going back to the choice phase---something akin to abandoning a fruitless chase and directing effort elsewhere in a natural setting.

This variation was tested by Fortes and col. (\citeyear{Fortes2017}) employing pigeons, and it was found that they consistently use the escape response to end unreinforced trials. However, their preference did not change since prematurely ending trials increases the reinforcement rate of the lower-density alternative, and they already preferred it. A critical test for the model would be the use of rats, which prefer the high-density alternative, in a similar procedure. By escaping unreinforced trials they could alter the reinforcement density of the low-density alternative and make it more valuable than the high-density one---given long enough terminal links.

This project will test rats in a variant of the suboptimal choice procedure that introduces an escape response that has the function of prematurely ending trials, which will be available during all terminal links. The subjects will be evaluated along a range of terminal-link lengths, since a mathematical analysis suggests that the longer the terminal links are, the more advantageous it will be to escape them. It is expected that rats will emit this response preferentially during negative, signaled terminal links; and, at long-enough terminal links, their preference will revert from the non-discriminative, high density alternative, to the discriminative, low-density one.


\printbibliography

\end{document}
